\documentclass[10pt]{article}

\usepackage[a4paper]{geometry}

% \usepackage{amsmath}
% \usepackage{amssymb}
% \usepackage{caption}
% \captionsetup{justification=centering}
% \usepackage{subcaption}
\usepackage{hyperref} % for href
\usepackage{graphicx}
\usepackage{float}
% \usepackage{ragged2e} % for flushleft

\usepackage[table]{xcolor} % Required for \cellcolor

\title{\textbf{i281 CPU} \\
A Course Project for EE224 Digital Systems}

\author{
  Team Name: \textbf{googoogaga} \\
  Team Members: \\
  Visharad Srivastava (24B1202) \\
  Shridhar Patil (24B1261) \\
  Jai Bellare (24B1307)
}

\date{Submission Date: \textbf{23 November, 2025}}

\begin{document}

\maketitle

\vspace{2cm}

\newpage

\tableofcontents

\newpage

\section{Introduction}

In this report, we present our design, Verilog implementation, and simulation of the i281, an 8-bit CPU. This CPU was made for the course project of EE224 Digital Systems at IIT Bombay, instructed by Prof. Sachin Patkar.
We follow the work of \href{https://www.ece.iastate.edu/~alexs/classes/2024_Fall_2810/}{Prof. Alexander Stoytchev} and his group at Iowa State University.

\section{Architecture}

\begin{center}
  \includegraphics[width=\textwidth]{01_architecture_block_diagram.png}
\end{center}

The CPU is of the Harvard-style architecture, with separate code and data memories:
\begin{enumerate}
  \item \textbf{Code Memory} : 64 rows of 16-bit registers
  \item \textbf{Data Memory} : 16 rows of 8-bit registers
\end{enumerate}

\end{document}
